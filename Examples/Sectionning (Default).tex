\documentclass[precis,colourall]{frenchlaw}

% This is a sample file for <frenchlaw/cls> class and associated files. It is aimed at showcasing possible outputs of that class.

%\subprecis

\usepackage{lipsum}

\begin{document}

% \makeatletter
% \show\l@heading
% \makeatother

\tableofcontents*

\shortcontents*

\book{Exemple de livre I}

\book{Exemple de livre II}

\part{Exemple de partie I}

\part{Exemple de partie II}

\heading{Exemple de titre I}

\heading{Exemple de titre II}

\subheading{Exemple de sous-titre I}

\subheading{Exemple de sous-titre II}

\chapter*{Starred Chapter}

\para Non numéroté

\fakechapter{Introduction}

\para Non numéroté

\section{Test}

\fakechapter[Avant propos]{Titre de l'avant-propos}

\chapter{Exemple de chapitre I}

\section{Exemple de section}

\chapter{Exemple de chapitre II}

\para Non numéroté

\para[Titre du premier paragraphe coulé] \lipsum[1]

\subpara[A subpara]

\subpara[A subpara too]

\para[Titre du deuxème paragraphe coulé] \lipsum[2]

\para[Titre du troisième paragraphe coulé] \lipsum[3]

\section{Exemple de section}

\para[Titre du quatrième paragraphe coulé] \lipsum[4]

\para[Titre du cinquième paragraphe coulé] \lipsum[5]

\para[Titre du sixième paragraphe coulé] \lipsum[6]

\section{Exemple de section}

\section{Exemple de section}

\subsection{Exemple de sous-section}

\subsubsection{Exemple de sous-sous-section}

\paragraph{Exemple de paragraphe}

\subparagraph{Exemple de sous-paragraphe}

\chapter{Exemple de chapitre III}

\section{Exemple de section}

\end{document}
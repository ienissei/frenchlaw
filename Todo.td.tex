
.bbx --- {article} {book} {collection} {proceedings} {inbook} {incollection} {inproceedings} {thesis} + {crossref}



Mettre une espace insécable avant les références juridiques (cf. citep)

Abréger les op. cit. pour certains types (?)
Utiliser un tracker arrêt précité et des commentaires au format: n. sous arrêt précité (?)

Neutraliser les op. cit., etc. sur les (in)reference (ajouter un \ifentrytype dans les macros qui placent les abréviations, pour imprimer la {posnote})

Les références qui contiennent un \cite sont parfois incompatibles avec \citet, mais fonctionnent sur les modes standard. Documenter.

Pas d'espace avant la ponctuation haute si anglais et juste après un \citet. Supprimer language=auto ou trouver une autre solution? Revoir la définition d'origine...

Voir \citereset (biblatex1.sty) pour les commandes de reset des trackers:
	\blx@ibidreset@force
	\blx@idemreset@force
	\blx@opcitreset@force
	\blx@loccitreset@force
> Pour les divisions jusqu'au chapitre. Il y a une option (à trouver) qui \citereset à chaue chapitre -- modifier pour inclure les subdivisions supérieures et limiter le reset aux seuls trackers.


{artice}
	- Supprimer numéro de page en doublon (?)

{thesis}
	- Abréger 'Th.'

{legislation}
	- Manque {du / de la + auteur} après le numéro (pour le droit européen). Utiliser {organization} (?) ou inverser avec {institution}.
	- Dans le .cbx, utiliser un 'précité(e)' s'il y a des {related} (cf. jurisdiction)

{jurisdiction}
	- Remplacer {type} (?)
	- \nocite (ou cite) des related, dans le .cbx
	- Mettre la post-note avant les revues (et {series}), dans le .cbx
	- Utiliser un *précité* dans le .cbx s'il y a des {related}, citer la référence avec \settoggle{bbx:related}{false}

{commentary}
	- Remplacer {type} (?)

Voir pour les initiales du prénom et éventuellement les premières consonnes (?)


Biber
»	- Type = jurisdiction et Institution = CC, CConst, CJUE, CJCE, CEDH (ou texte développé), ajouter automatiquement le texte comme keyword dans biber
»	- Programmer biber pour ajouter les "du" et "de la" pour le .cbx via {note}


Reproduire un {yyyy-mm-dd} pour le JO, JOUE, etc. e.g. \JO{2015-10-22} ou \JOUE{Serie + date + page}

Remplacer {hyphenation} par {langid} dans les bibliographies et le code. Sinon les champs hérités type crossref ne le voient pas. Et mettre {french} par défaut dans biber, s'il n'y a pas d'indication.

Pour les entrées complexes, e.g. un traité avec la loi de ratification, utiliser une autre entrée avec keyword=noprint, et un appel avec \cite{X} dans le champ {note} de l'entrée principale. Idem pour les arrêts non publiés, rapportés dans une revue, dans le champ {series}. Il faut compiler avec biber + xelatex + biber.

Keyword == noprint pour les références qui ne sont pas en bibliographie générale. E.g. une loi de ratification d'un traité ou la référence à un JO (à créer)

Support matériel == entre crochets, dans le {titleaddon} (?). Ou créer un champ



# TODO #

Voir les styles: chicago, historian, philosophy

Shorttitle		== Supprimer les doublons (cf. 3.6.4) + Shorttitle en italiques + Sorting
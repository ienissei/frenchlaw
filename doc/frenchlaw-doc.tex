\documentclass[oneside,marginal]{frenchlaw}

%\usepackage{snapshot}

\usepackage{frenchlaw-dev}
%\usepackage{Typographie}

%	Page de titre <dev>
\title{La classe Frenchlaw}
\subtitle{ \unskip\LaTeX\ à l'usage des juristes}
\author{Flora \textsc{Vern}}
\email{ienissei@gmail.com}

\addcs{@,setmainfont,setlength,dimexpr,addto@X,maketitleX,maketitle,titlepagestyle,@title,aliaspagestyle,book,part,heading,subheading,chapter,section,subsection,subsubsection,paragraph,subparagraph,para,sidebar,footmarkwidth,footmarksep,footparindent,symbolfootnote,resetfootnotes,notespacing}

\begin{document}

\makeatletter
\let\frontmatter=\@smemfront
\makeatother

\maketitle

\cleardoublepage
\tableofcontents*

\chapter{Introduction}

La classe |frenchlaw| est destinée à fournir aux juristes français un ensemble d'outils pertinents pour leur travail d'écriture, universitaire ou praticien. 

Elle est basée sur la classe |memoir| qui propose de nombreuses fonctionnalités, généralement compatibles avec |frenchlaw|. Ces dernières ne seront pas documentées ici si elles ne présentent pas un intérêt immédiat ou n'ont pas fait l'objet de modifications spéciales. Il peut donc être utile de consulter la documentation spécifique de |memoir|.

La présente documentation détaille les nouveautés apportées par |frenchlaw|, tout en s'efforçant de fournir également une introduction à \LaTeX; de nombreux manuels et tutoriels sont disponibles dans les distributions standard% et sur: http://ctan.org/
.\todo{URL}

\section{Explication d'un document}

Pour créer un document simple, il suffit de mettre le texte suivant dans un fichier |.tex| et de compiler avec \LaTeX\ ou \XeLaTeX:

\begin{macro}[frenchlaw,Ceci,est,un,document]
\documentclass[<options>]{frenchlaw}
 
% Préambule
 
\begin{document}
 
% Corps du document
Ceci est un document \TeX.
 
\end{document}
\end{macro}

Ici et partout ensuite, les éléments colorés commençant par un |backslash|~(|\|) sont dénommés commandes, déclarations ou~-- de manière générique~-- macros.

Les éléments placés entre |[crochets]| sont optionnels et peuvent être omis. Les éléments entre |<chevrons>| doivent être remplacés par une valeur indiquée dans la documentation~-- ici, |<options>| sera remplacé par le nom de l'une des options décrites au chapitre suivant.

On appelle |préambule| tout ce qui se trouve avant la déclaration |\begin{document}| et |document| tout ce qui se trouve après celle-ci. Aucun texte ou commande n'est autorisé après |\end{document}|. Le contenu additionnel du préambule sera expliqué pour chaque type de document.

Les éléments en rouge et précédés d'un |%|
sont des commentaires, qui dans cette documentation indiquent généralement la valeur par défaut donnée à une commande.

%\section{Installation}
%\todo{À écrire}

%\section{Erreurs fréquentes}
%\todo{À écrire} % compiler deux fois, underfull et overfull boxes, undefined commands, etc.

\mainmatter*

\chapter{Options et dépendances}

La classe |frenchlaw| a été conçue avec de nombreuses options\footnote{Plusieurs options peuvent être sélectionnées; elles sont alors séparées d'une virgule. En cas d'options mutuellement exclusives, seule la dernière sera prise en compte; en cas d'options contradictoires, l'une d'entre-elles sera automatiquement désactivée.} qui permettent, très facilement, de paramétrer l'apparence générale du document.

\section{Options générales}

Les options spécialement prises en compte par |frenchlaw| sont les suivantes:

\begin{option}{a4paper, a5paper, a6paper}{a4paper}
Dimensions de la feuille de papier à utiliser. Contrairement aux options éponymes de la classe |memoir|, la taille du texte, les marges et le bloc de texte sont automatiquement définis par ces options.
\end{option}

\begin{option}{10pt, 12pt}{12pt}
Taille de la police d'écriture pour le corps du texte~(|12pt| par défaut, sauf |10pt| avec |a6paper|). Pour être effective, cette option doit être utilisée après celle définissant la taille du papier.
\par	Les valeurs recommandées pour une lisibilité optimale sont aux alentours de~65 caractères par ligne; ce ratio est très difficile à obtenir pour les tailles |a4| et |a6|, mais est optimisé pour les documents |a5|~(proches de la taille de nombreux livres), soit par défaut:
\begin{itemize}
\item Options |a4paper| et |12pt|~(défaut) \hfill 92 caractères par ligne
\item Options |a5paper| et |12pt| \hfill 65 caractères par ligne
\item Options |a6paper| et |10pt| \hfill 57 caractères par ligne
\end{itemize}
\end{option}

\begin{option}{oneside, twoside}{twoside}
Préparer le document pour une impression en recto-verso par défaut~(les marges ne sont pas identiques à gauche et à droite), ou bien pour une impression en recto simple~(marges identiques).
\end{option}

\begin{option}{onecolumn, twocolumn}{onecolumn}
Division optionnelle de la page en deux colonnes pour une meilleure lisibilité. Cette option n'est pas recommandée avec les petites tailles de papier, pour des raisons d'esthétique. En |a4|, il est préférable d'utiliser une police de |10pt| pour les mêmes raisons. L'option |twocolumn| désactive automatiquement |marginal|.
\end{option}

\begin{option}{linespacing}[value]{1.1}
L'option |linespacing| permet de d'ajuster l'interligne par rapport aux valeurs standard, en redéfinissant en interne la valeur de |\baselinestretch|. La classe est fournie avec des lignes d'une hauteur de |10pt/12pt| ou |12pt/14.5pt|, soit un interligne de base de $1,2$. En donnant par défaut |linespacing=1.1|, on obtient ainsi des lignes de |10pt/13.2pt| ou |12pt/16pt|, soit un interligne équivalent à $1,33$.
\par	Cette valeur donne de bons résultats visuels et ne devrait pas être modifiée sans raison, sauf éventuellement au profit de l'interligne médian, pour un texte plus aéré. Pour référence, les valeurs couramment demandées pour les manuscrits sont obtenues comme suit:
\begin{itemize}
\item Interligne simple (exactement égal au corps du texte) \hfill |linespacing=0.83|
\item Interligne médian (une fois et demie le corps du texte) \hfill |linespacing=1.25|
\item Interligne double (soit l'équivalent d'un saut de ligne) \hfill |linespacing=1.66|
\end{itemize}
\end{option}

\begin{option}{article}{false}
Cette option permet d'alléger les subdivisions hiérarchiques; elle est particulièrement adaptée aux textes brefs. Elle est activée automatiquement par toutes les options spécifiques à des textes courts, et ne devrait donc être explicite que pour les articles universitaires.
\par	En mode |article|, les chapitres (|\chapter|) ne commencent pas sur une nouvelle page et, à partir de la |\section|, la numérotation est devient la suivante: I, A, 1, a.
\par	Par opposition, le mode normal se présente traditionnellement par: Section~I, \S~1, A, 1, a.
\end{option}

\begin{option}{marginal}{false}
Cette option permet d'augmenter la taille de l'une des marges latérales: à droite pour les documents en mode |oneside| et en alternance en mode |twoside|. La marge est alors utilisée pour imprimer les titres des subdivisions secondaires~(|\paragraph|, |\subparagraph|, et |\para| si |precis| est activé). Elle est incompatible avec l'option |twocolumn|.
\end{option}

\begin{option}{precis}{false}
Cette option permet d'imprimer le nom donné aux paragraphes numérotés avec la commande |\para| et de l'insérer dans la table des matières avec le numéro de page correspondant. Cela revient à automatiser la création d'un précis pour chaque section ou chapitre.
\end{option}

\section{Options spéciales}

D'autres options sont disponibles pour créer des types de documents plus spécifiques; leur fonctionnement fera l'objet d'une explication détaillée dans des sections ultérieures.

\begin{option}{didactic}{false}
Cette option présente le texte sous une forme plus lisible, en deux colonnes, avec des informations dans l'entête et le pied de chaque page. Elle est adaptée pour faire des présentations ou des fiches brèves.
\par	Elle charge automatiquement les options: |twocolumn|, |oneside|, |10pt| et |article|, mais il est possible de charger d'autres options après pour altérer ce comportement.
\end{option}

\section{Options de language}

|Frenchlaw| est prévue pour charger automatiquement les scripts suivants: |grec|, |latin|, |anglais| (US et UK) et |français|, en utilisant le français comme langue principale du document. Ces paramètres peuvent être modifiés à l'aide des options suivantes:

\begin{option}{british, english, french}{french}
Choix de la langue principale du document pour le package |babel| et tous les autres packages qui en dépendent. Les langues prises en charge sont le français et l'anglais, avec une variante pour la graphie britannique ou américaine.
\end{option}

\begin{option}{italian, german, spanish}{aucune}
Les langues disponibles dans |babel| peuvent être chargées en supplément des langues principales, et serviront toujours comme langues secondaires~-- utilisables pour un extrait, une citation, une référence bibliographique. Seules les langues mentionnées dans la documentation sont prises en charge~-- mais une mise à jour peut être effectuée sur demande sans difficulté.
\end{option}

\section{Dépendances}

La classe |frenchlaw| utilise en interne un certain nombre de packages avec pour conséquences notables:
\begin{itemize}
\item que les commandes et déclarations offertes par ces packages externes sont disponibles directement, en respectant la documentation fournie;
\item que des conflits et erreurs peuvent survenir sur des fonctionnalités qui n'ont pas été testées, ou bien après une mise à jour de ces dépendances: dans les deux cas, merci de bien vouloir les signaler au développeur responsable de |frenchlaw|.
\end{itemize}

L'utilisation de cette classe requiert le format \LaTeXe\ et l'extension \eTeX\, qui sont standard dans les distributions. Les dépendances principales de la présente classe sont les suivantes: \todo{Compléter la liste.}%
|memoir| (classe), |babel|, |csquotes|, |fixltx2e|, |fixltxhyph|, |xkeyval|.

Les polices d'écriture sont chargées avec les packages standard: |inputenc| et |fonctenc| sous \LaTeX, et |fontspec| et |xltxtra| sous \XeLaTeX. Le package |microtype| est également utilisé par les deux moteurs pour un réglage typographique plus fin.

\section{Polices d'écriture}

Les polices d'écriture utilisées pour le corps du texte sont |Linux Libertine| sous \LaTeX\ et |Junicode| sous \XeLaTeX; elles sont complémentées par la police |Palatino| pour les expressions mathématiques, et par |Latin Modern| pour les versions sans-serif et à chasse-fixe.

Ce choix permet de n'utiliser que des polices d'écriture libres, présentes dans les distributions standard et ne requérant pas d'installation supplémentaire. Elles ont par ailleurs été sélectionnées pour leurs lignes classiques, leur grande qualité typographique et la palette de caractères qu'elles proposent~-- notamment touts les caractères latins accentués et l'alphabet grec, ainsi que le cyrillique et l'hébraïque pour |Linux Libertine|, et les scripts médiévaux pour |Junicode|.

Il est possible de leur substituer d'autres polices d'écriture selon les méthodes habituelles, y compris des fontes commerciales sous \XeLaTeX. Grâce à |fontspec|, ce moteur permet de gérer toutes les options des polices OpenType.

%\chapter{Personnalisation}
%\todo{À écrire}
% Personnalisation via le préambule ou via un fichier .conf
% Lister toutes les commandes de personnalisation

% \resetfootnotes (per page)

\part{Guide pour le formatage des documents}

\chapter{Divisions logiques des documents}

\LaTeX\ permet de standardiser le style des divisions logiques présentes dans un document: il existe une commande pour chaque division logique; ce sont ces commandes qui gèrent la numérotation, le formatage et l'insertion dans la table des matières des divisions du document. En droit, on utilise des divisions et subdivisions hiérarchiques~(chapitre, section,~etc.) ainsi que des divisions autonomes (paragraphe, clause,~etc.). Dans les monographies, il est également d'usage d'employer des divisions propres au volume, qui permettent de distinguer les pages liminaires, le corps du texte, et les références.

\section{Les divisions hiérarchiques}

Par défaut, \LaTeX\ utilise un système de divisions logiques qui a été ici repris et adapté pour le droit. Ces divisions étant hiérarchisées, l'utilisation d'une division supérieure entraîne la réinitialisation de la numérotation des divisions qui lui sont inférieures.

Les deux premières commandes prennent une pleine page~(recto) pour insérer le titre et le numéro des livres et parties; les deux commandes suivantes insèrent les titres et sous-titres sur une nouvelle page~(recto).

\begin{macro}
\book{<titre>}% Livre premier, Livre II
\part{<titre>}% Première partie, Deuxième partie
\heading{<titre>}% Titre premier, Titre deuxième
\subheading{<titre>}% Sous-titre premier, Sous-titre deuxième
\end{macro}

La division suivante crée des chapitres, qui sont sont automatiquement insérés sur une nouvelle page~(recto), sauf si l'option |article| est activée. Le chapitre est considéré comme la division standard par \LaTeX; la numérotation des notes de bas de page et des figures y sont subordonnées. Elle est généralement utilisée exactement comme les autres, soit:

\begin{macro}
\chapter{<titre>}% Chapitre premier, Chapitre II
\end{macro}

Cependant, cette division offre des options et une variante supplémentaire:

\begin{macro}
\chapter[<titre-toc>][<en-têtes>]{<titre>}
\chapter*[<en-têtes>]{<titre>}
\end{macro}

La différence entre |\chapter| et sa variante étoilée réside dans ce que la seconde crée un chapitre non-numéroté et absent de la table des matières, très utile pour les introductions. Le chapitre étoilé ne réinitialise aucun compteur, ce qui oblige à le faire manuellement si besoin. Il est également possible de rajouter manuellement celui-ci à la table des matières en le faisant directement suivre du code suivant:

\begin{macro}[toc,chapter]
\addcontentsline{toc}{chapter}{<titre-toc>}
\end{macro}

Dans les deux cas, l'argument |<titre-toc>| contient le titre tel qu'il apparaît dans la table des matières; l'argument |<en-têtes>| contient le titre qui est imprimé en haut de chaque page~(verso). Si rien n'est spécifié, |\chapter| utilisera le nom du chapitre dans la table des matières et les en-têtes, alors que |\chapter*| n'imprimera rien.

Les divisions suivantes créent des subdivisions diverses; elles ne créent pas de nouvelle page et sont numérotées par |frenchlaw| suivant l'usage juridique. Ces éléments sont affectés par différentes options, notamment |article|~(qui allège la numérotation pour les textes courts) et |marginal|~(qui place les deux dernières subdivisions dans les marges). Les divisions sont les suivantes, avec leur numérotation normale et en mode |article|~(parenthèses):

\begin{macro}
\section{<titre>}% Section I. (ou: I.)
\subsection{<titre>}% § 1. (ou: A.)
\subsubsection{<titre>}% A. (ou: 1.)
\paragraph{<titre>}% 1. (ou: a.)
\subparagraph{<titre>}% a. (ou vide)
\end{macro}

Toutes ces divisions acceptent les mêmes options et variantes, limitées dans ce cas à un titre différent pour la table des matières et une version étoilée non-numérotée et absente de la table des matières, soit par exemple:

\begin{macro}
\section[<titre-toc>]{<titre>}
\section*{<titre>}
\end{macro}

Pour insérer une section étoilée dans la table des matières, il est possible de la faire suivre du code suivant, où |<section>| correspond à l'intitulé de la division (|section|, |subsection|, |subsubsection|, |paragraph| ou |subparagraph|):

\begin{macro}[toc]
\addcontentsline{toc}{<section>}{<titre-toc>}
\end{macro}

Toutes les divisions hiérarchiques ont été créées dans le respect des commandes éponymes de la classe |memoir|. Les |\heading| et |\subheading| reprennent les mêmes éléments stylistiques, et peuvent donc être modifiés suivant les mêmes règles. La documentation de |memoir| contient toutes les informations nécessaires concernant la numérotation et le formatage des divisions hiérarchiques.

\section{Les divisions autonomes}% \para, \article, \clause, etc.

\section{Les divisions de volume}% \frontmatter / \mainmatter / \appendix (?) \ \backmatter

\chapter{Éléments généraux de formatage des documents}

La classe |frenchlaw| prend en charge de nombreux paramètres afin de produire des documents d'une qualité typographique satisfaisante, compatible tant avec les usages modernes qu'avec les normes typographiques d'usage en France et dans les milieux juridiques.

Cela implique des choix quant au format des pages~(d'inspiration classique), à la taille du texte et des espacements verticaux~(application stricte d'une grille typographique pour que les lignes du recto et du verso de la feuille se superposent exactement), et aux règles appliquées par \LaTeX\ dans le découpage des lignes, des paragraphes et des pages.

La combinaison de toutes ces contraintes suppose que l'on exige une plus grande rigueur typographique pour les documents universitaires et les rapports où l'on trouvera plus régulièrement des paragraphes longs et des références de bas de page, et une rigueur moindre en ce qui concerne les documents très techniques, rédigés dans un style différent.\todo{Régler les pénalités pour les docs techniques}

Ces différences sont prises en compte par |frenchlaw|, mais il appartient en dernière analyse à l'auteur de supprimer ou rajouter un mot pour «~gagner~» ou «~perdre~» une ligne et harmoniser ainsi les paragraphes ou les pieds de page.

\section{Commandes génériques}
%\todo{Commandes de base, type \emph, \latin, etc.}

\section{Abréviations communes}
%\todo{Abréviations}

\section{Notes de page et de marge}

Les différents types de notes ont été reprises à la classe |memoir|, avec quelques adaptations. Il est donc possible d'obtenir des notes de bas de page classiques~(numérotées) ou bien symbolisées par une obèle (\textdagger), par exemple pour les remerciements d'un article.

\begin{macro}
\footnote{<Texte de la note>} % Note de bas de page classique
\symbolfootnote{<Texte de la note>} % Note signalée par une obèle
\end{macro}

Ces commandes s'utilisent à l'endroit où doit apparaître l'appel de note, et sont ensuite imprimées au bas de la page. Il est possible d'ajouter |\par| à la fin d'une |\symbolfootnote| afin de forcer un saut de ligne entre celle-ci et la note de bas de page numérotée qui suivrait.

Par défaut, la numérotation est remise à zéro à chaque chapitre; il est possible d'imposer une numérotation par page en insérant |\resetfootnotes| dans le préambule ou le fichier de configuration.

Les notes de bas de page ne respectent pas l'alignement vertical du document afin de ne pas imposer un interligne trop grand, mais elles sont toutes propulsées en bas de page et sont donc alignées entre elles.

De manière générale, sauf dans des cas très particuliers déjà pris en charge par la classe, il convient de préférer les |\footnote| aux autres types de notes, car elles constituent la principale variable d'ajustement typographique de |frenchlaw|.

Des notes marginales~-- brèves et peu nombreuses afin d'éviter les déplacements et erreurs~-- peuvent être produites, de préférence en conjonction avec l'option |marginal|:

\begin{macro}
\marginpar{<Texte de la note>} % Note insérée face à son point d'appel
\end{macro}

Pour insérer de longs commentaires dans les marges, y compris sur plusieurs pages consécutives, une autre commande existe qui ne tient pas compte de l'alignement vertical et imprime toutes les notes à la suite les unes des autres à partir du haut de la page:

\begin{macro}
\sidebar{<Texte de la note>} % Note sous forme de paragraphe
\end{macro}

Les notes marginales ne sont pas numérotées ni signalées dans le texte, car leur usage ne s'y prête pas; elles servent d'explication ponctuelle~(|\marginpar|) ou de complément général d'information~(|\sidebar|).

\section{Listes et énumérations}

\section{Citations et épigraphes}

\section{Tableaux et figures}

\chapter{Éléments pour le formatage de documents juridiques}

\section{Création et utilisation d'un clausier}

\section{Inclusion de pièces annexes et documents externes}

%\part{Guide pour les documents juridiques spécifiques}
%
%\chapter{Monographies, thèses, rapports}
%% Inclure la question du \fontmatter, \mainmatter, \backmatter
%
%\chapter{Articles, dissertations, commentaires}
%
%\chapter{Fiches, documents pédagogiques}
%
%\chapter{Contrats}
%
%\chapter{Conclusions}
%
%\chapter{Assignations}
%
%\chapter{Lettres et mémos}
%
%\chapter{Curriculum vitæ}
%
%\part{Guide pour les bibliographies}

\end{document}